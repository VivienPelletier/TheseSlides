\begin{frame}{Stratégie innermost}
  \begin{itemize}
  \item Restriction des réécritures
  \item Réécritures de sous-termes non réductibles
  \end{itemize}
  \begin{center}
    \begin{tikzpicture}
      
      \tikzstyle{focusedNode} = [shape = circle, thick, focusColor, fill=focusColor, draw];
      \tikzstyle{normalEdge} = [thick];
      
      \draw[thick, focusColorBis, fill=focusColorBis] (1,-3) -- ( 0,-5) -- (2,-5) -- cycle;
      
      \node (root) at (0,0) {};
      
      \node (a)    at (-2,-1) {};
      \node (b)    at ( 0,-1) {};
      \node (c)    at ( 2,-1) {};

      \node (aa)  at (-3,-2) {};
      \node (ab)  at (-2,-2) {};
      \node (ba)  at (-1,-2) {};
      \node (bb)  at (-0,-2) {};
      \node (bc)  at ( 1,-2) {};
      
      \node (bba)  at (-1,-3) {};
      \node[focusedNode] (bbb)  at ( 1,-3) {};

      \draw[normalEdge] (root) -- (a);
      \draw[normalEdge] (root) -- (b);
      \draw[normalEdge] (root) -- (c);

      \draw[normalEdge] (a) -- (aa);
      \draw[normalEdge] (a) -- (ab);
      \draw[normalEdge] (b) -- (ba);
      \draw[normalEdge] (b) -- (bb);
      \draw[normalEdge] (b) -- (bc);
      
      \draw[normalEdge] (bb) -- (bba);
      \draw[normalEdge] (bb) -- (bbb);
    \end{tikzpicture}
  \end{center}
\end{frame}

\begin{frame}{Irréducibilité forte}
  \begin{Definition}[Irréducibilité forte]
    Pour tout $l \rightarrow r \in R$.\\
    \begin{itemize}[<+->]
    \item Un terme $t$ est {\em fortement irréducible} (par $R$) si\\
      aucun de ses sous-termes ne peut s'unifier avec $l$.\\
    \item Une substitution $\theta$ est {\em fortement irréducible} si\\
      pour tout $x \in \mathcal{X}$, $\theta(x)$ est fortement irréducible.
    \end{itemize}
  \end{Definition}
  \begin{itemize}
  \item \onslide<3-> $f(b) \rightarrow b$
  \item \onslide<4-> $t = f(x)$ : \onslide<5-> irréducible
  \item \onslide<6-> $\theta = (x/a)$ : \onslide<7-> est fortement irréducible
  \item \onslide<8-> $\theta(t) = f(a)$ : \onslide<9-> fortement irréducible
  \end{itemize}
\end{frame}

\begin{frame}{{\large Dérivations non copiantes (\nc) et fortement non copiantes (\snc)}}
  \begin{Definition}
    Soit $A$ un atome ($A$ peut contenir des variables).\\
    \begin{itemize}[<+->]
    \item L'étape $A \leadsto_{[H\leftarrow B,\sigma]} G$ est \nc (resp.\ \snc) si \\
      pour toute variables $x \in Var(H)$ non linéaires,\\
      $x$ est irréducible (resp.\ fortement irréducible) par $R$.\\
    \item Une dérivation est \nc (resp.\ \snc) si toutes ses étapes le sont.
    \end{itemize}
  \end{Definition}
  \begin{itemize}[<+->]
  \item Une clause non copiante est \snc
  \end{itemize}
  \onslide<4->
  \begin{exampleblock}{Exemple}
    \begin{itemize}[<+->]
    \item $P(g(x,x)) \leftarrow Q(x)$
    \item $R = \{h(a) \rightarrow b\}$
    \item $P(g(h(y),h(y)))$ \onslide<7->$\leadsto Q(h(y))$ : \onslide<8->\nc mais pas \snc
    \end{itemize}
  \end{exampleblock}
\end{frame}

\begin{frame}{Clôture par réécriture innermost}
  \begin{alertblock}{Théorème [WRLA, 2016, Y. Boichut, V. Pelletier et P. Réty]}
    Soit $R$ un système de réécriture linéaire gauche.\\
    \pause
    Soit $Prog$ un \csprogramme normalisé et non copiant.\\
    \pause
    Soit $Prog' \supseteq Prog$ un \csprogramme \\
    \pause
    ~~~tel que toutes les paires critiques de $Prog$ sont convergentes\\
    ~~~par dérivations \snc dans $Prog'$.\\
    \pause
    Si un terme $t \in \mathcal{L}_{Prog}(P)$ et $t \rightarrow^*_R t'$ avec une stratégie innermost, alors $t' \in \mathcal L_{Prog'}(P)$.
  \end{alertblock}
\end{frame}
