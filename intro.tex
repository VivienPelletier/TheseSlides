\begin{frame}{Systèmes complexes}
  \begin{block}{Système complexe}
    Un système complexe est un ensemble constitué d'un grand nombre d'entités en interaction qui empêchent l'observateur de prévoir sa rétroaction, son comportement ou évolution par le calcul.
  \end{block}
  \pause
  \begin{exampleblock}{Systèmes complexes}
    \begin{itemize}
    \item programmes informatiques,
    \item protocoles de sécurité,
    \item circuits logiques \ldots
    \end{itemize}
  \end{exampleblock}
\end{frame}

\begin{frame}{Le génie logiciel}
  \begin{columns}
    \begin{column}{0.6\textwidth}
      \begin{center}
        \includegraphics[height=4cm]{media/apollo.png}
      \end{center}
      \onslide<2>
      \begin{block}{Génie logiciel}
        L'ensemble des activités de conception et de mise en œuvre des produits et des procédures tendant à rationaliser la production du logiciel et son suivi.
      \end{block}
    \end{column}
    \begin{column}{0.4\textwidth}
      \onslide<1->
      \begin{center}
        \includegraphics[width=0.9\textwidth]{media/MHamilton.jpg}
      \end{center}
    \end{column}
  \end{columns}
\end{frame}

\begin{frame}{Vérification}
  Conformité et fiabilité d'un système complexe ?
  \pause
  \begin{block}{}
    \begin{itemize}[<+->]
    \item Les tests
      \begin{itemize}
      \item si l'ensemble d'entrées est trop grand ou infini ?
      \end{itemize}
    \item Les méthodes formelles
      \begin{itemize}
      \item analyse statique par interprétation abstraite
      \item vérification déductive
      \item vérification de modèles
      \end{itemize}
    \end{itemize}
  \end{block}
\end{frame}

\begin{frame}{Analyse statique par interprétation abstraite}
  \begin{columns}
    \begin{column}{0.3\textwidth}
      \begin{center}
        \includegraphics[width=0.5\textwidth]{media/fichier.png}
        \includegraphics[width=\textwidth]{media/entonnoir.png}
        \includegraphics[width=0.25\textwidth]{media/fichier.png}
      \end{center}
    \end{column}
    \begin{column}{0.5\textwidth}
      \begin{block}{}
        \begin{itemize}
        \item Constat
          \begin{itemize}
          \item trop d'informations
          \end{itemize}
        \item Solution
          \begin{itemize}
          \item utilisation d'abstractions
          \end{itemize}
        \item Difficulté
          \begin{itemize}
          \item garder suffisamment d'informations
          \item mais pas trop
          \end{itemize}
        \end{itemize}
      \end{block}
    \end{column}
  \end{columns}
\end{frame}

\begin{frame}{Vérification déductive}
  \begin{overprint}
    \begin{tikzpicture}
      
      \tikzstyle{domino} = [minimum width=1.7cm, minimum height=1cm, draw]
      \tikzstyle{dependancy} = [->, thick]
      
       \onslide<1->{
         \node (shredder) at (0,1.5) {\includegraphics[width=0.3\textwidth]{media/shredder.png}};
       }

      \onslide<2->{
        \node[circle, minimum size=0.5cm, very thick, fill=white, draw] (small) at (0,0.4) {};
        \node[circle, minimum size=7cm, very thick, draw] (big) at (6,-0.5) {};
      
        \node[domino] (dom3pre) at (6, -2) {\tiny{préconditions}};
        \node[domino, below=0 of dom3pre] (dom3post) {\tiny{postconditions}};
        
        \draw[very thick] (small) -- (big);
      }
      
      \onslide<3->{
        \node[domino] (dom1pre) at (4.75, 1.25) {\tiny{préconditions}};
        \node[domino, below=0 of dom1pre] (dom1post) {\tiny{postconditions}};
      
        \node[domino] (dom2pre) at (7.25, 1.25) {\tiny{préconditions}};
        \node[domino, below=0 of dom2pre] (dom2post) {\tiny{postconditions}};
      
        \draw[dependancy] (dom1post.south) |- (5.5,-0.75) -- (dom3pre.134);
        \draw[dependancy] (dom2post.-134) |- (6.5,-0.75) -- (dom3pre.46);
        \draw[dependancy] (dom2post.-46) |- (8.25,-0.75) -- (8.25,2) -| (dom2pre.46);
        
        \draw[dependancy] (big.66) -| (dom2pre.134);
        \draw[dependancy] (big.north) -- (6,2.5 ) -| (dom1pre.north);
      }
    \end{tikzpicture}
  \end{overprint}
\end{frame}

\begin{frame}{Vérification de modèles}
  \begin{block}{}
    \begin{itemize}
    \item Analyse exhaustive
    \item Représentation astucieuse
    \end{itemize}
  \end{block}
  \begin{columns}
    \begin{column}{0.5\textwidth}
      \includegraphics[width=\textwidth]{media/ftdi.jpg}
    \end{column}
    \begin{column}{0.5\textwidth}
      \includegraphics[width=\textwidth]{media/sftdi.jpg}
    \end{column}
  \end{columns}
\end{frame}

\begin{frame}{Analyse d'accessibilité}
  Configurations
  \begin{block}{}
    \begin{itemize}[<+->]
    \item Ensemble des configurations accessibles
    \item Ensemble des configurations indésirables
    \end{itemize}
  \end{block}
  \begin{columns}
    \begin{column}{0.45\textwidth}
      \onslide<3->
      \begin{center}
        \begin{tikzpicture}
          \draw[fill=accessibleColor, fill opacity=0.5] (0,0) ellipse (1 and 0.75);
          \node (r*i) at (0,0) {\scriptsize{Accessibles}};
          \draw[fill=badColor, fill opacity=0.5] (1.75,0) ellipse (1 and 0.75);
          \node (bad) at (1.75,0) {\scriptsize{Indésirables}};
        \end{tikzpicture}\\
        Il existe une configuration\\ indésirable accessible
      \end{center}
    \end{column}
    \begin{column}{0.45\textwidth}
      \onslide<4>
      \begin{center}
        \begin{tikzpicture}
          \draw[fill=accessibleColor, fill opacity=0.5] (0,0) ellipse (1 and 0.75);
          \node (r*i) at (0,0) {\scriptsize{Accessibles}};
          \draw[fill=badColor, fill opacity=0.5] (2.5,0) ellipse (1 and 0.75);
          \node (bad) at (2.5,0) {\scriptsize{Indésirables}};
        \end{tikzpicture}\\
        Aucune configuration indésirable\\ n'est accessible
      \end{center}
    \end{column}
  \end{columns}
\end{frame}

\begin{frame}{Calcul de l'ensemble des configurations accessibles}
  \begin{itemize}[<+->]
  \item Configurations initiales : $I$
  \item Dynamique : $R$
  \end{itemize}
  \begin{center}
    \begin{tikzpicture}
      \onslide<7->
      \draw[fill=fig-color-ri] (4,0) ellipse (5 and 2.2);
      \node (r*i) at (7.75,0) {$R^*(I)$};

      \onslide<6->
      \draw[fill=fig-color4] (3,0) ellipse (4 and 1.9);
      \node (...) at (5.75,0) {$\cdots$};
      
      \onslide<5->
      \draw[fill=fig-color3] (2,0) ellipse (3 and 1.6);
      \node (r2i) at (3.75,0) {$R^2(I)$};

      \onslide<4->      
      \draw[fill=fig-color2] (1,0) ellipse (2 and 1.3);
      \node (ri) at (1.75,0) {$R(I)$};

      \onslide<3->
      \draw[fill=fig-color-i] (0,0) ellipse (1 and 1);
      \node (i) at (0,0) {$I$};
      
      \onslide<1->
    \end{tikzpicture}
  \end{center}
\end{frame}

\begin{frame}{Sur-approximations}
  \begin{center}
    \begin{tikzpicture}
      \draw[fill=overApproxColor, fill opacity=0.5] (0,0) ellipse (2 and 1);
      \node (r*i) at (-1.25,0) {$\mathcal H$};
      
      \draw[fill=accessibleColor, fill opacity=0.5] (0,0) ellipse (1 and 0.75);
      \node (r*i) at (0,0) {\scriptsize{Accessibles}};
      \draw[fill=badColor, fill opacity=0.5] (3.5,0) ellipse (1 and 0.75);
      \node (bad) at (3.5,0) {\scriptsize{Indésirables}};
    \end{tikzpicture}\\
    Aucune configuration indésirable accessible\\
  \end{center}
  \begin{columns}
    \begin{column}{0.5\textwidth}
      \pause
      \begin{center}
        \begin{tikzpicture}
          \draw[fill=overApproxColor, fill opacity=0.5] (0,0) ellipse (2 and 1);
          \node (r*i) at (-1.25,0) {$\mathcal H$};
          
          \draw[fill=accessibleColor, fill opacity=0.5] (0,0) ellipse (1 and 0.75);
          \node (r*i) at (0,0) {\scriptsize{Accessibles}};
          \draw[fill=badColor, fill opacity=0.5] (1.75,0) ellipse (1 and 0.75);
          \node (bad) at (1.75,0) {\scriptsize{Indésirables}};
        \end{tikzpicture}\\
        Il existe une configuration\\ indésirable accessible
      \end{center}
    \end{column}
    \begin{column}{0.5\textwidth}
      \pause
      \begin{center}
        \begin{tikzpicture}
          \draw[fill=overApproxColor, fill opacity=0.5] (0,0) ellipse (2 and 1);
          \node (r*i) at (-1.25,0) {$\mathcal H$};
          
          \draw[fill=accessibleColor, fill opacity=0.5] (0,0) ellipse (1 and 0.75);
          \node (r*i) at (0,0) {\scriptsize{Accessibles}};
          \draw[fill=badColor, fill opacity=0.5] (2.5,0) ellipse (1 and 0.75);
          \node (bad) at (2.5,0) {\scriptsize{Indésirables}};
        \end{tikzpicture}\\
        Aucune configuration \\indésirable accessible,\\ c'est un faux-positif
      \end{center}
    \end{column}
  \end{columns}
\end{frame}

\begin{frame}{Notre modélisation}
  \begin{block}{}
    \begin{itemize}[<+->]
    \item Configuration : un terme
    \item Configurations initiales : un langage de termes
    \item Dynamique du système : un système de réécriture
    \item Configurations indésirables : un langage de termes
    \end{itemize}
  \end{block}
\end{frame}